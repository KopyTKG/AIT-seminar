\chapwithtoc{Historie}\label{chapter:predmluva}

Proxmox je oblíbená virtualizační platforma, která kombinuje virtuální servery (Proxmox Virtual Environment, PVE) a systém pro správu clusterů a virtualizovaného úložiště (Proxmox Virtual Environment Cluster). Její historie sahá do roku 2008, kdy byla poprvé uvedena na trh společností Proxmox Server Solutions GmbH, založenou v Rakousku.

Proxmox VE byl vytvořen jako open-source řešení, které poskytuje efektivní a snadno použitelné prostředí pro správu virtuálních serverů. Od začátku byl postaven na Linuxovém jádru a využíval virtualizační technologie jako KVM (Kernel-based Virtual Machine) a kontejnerovou technologii LXC (Linux Containers), což umožňovalo uživatelům vytvářet a spravovat virtuální stroje a kontejnery z jediného rozhraní.

V průběhu let došlo k několika důležitým vývojovým krokům. Významným milníkem bylo zavedení Proxmox VE Cluster, což umožnilo správu více serverů jako jednoho celku, což výrazně zjednodušilo správu zdrojů a redundanci. Další významnou aktualizací bylo zavedení Ceph, rozšířeného open-source úložného řešení, které umožňuje vysoce škálovatelné blokové úložiště.

Proxmox se také vyznačuje svým uživatelským rozhraním založeným na webovém prohlížeči, které umožňuje jednoduché správu a monitoring virtuálních strojů a úložiště. Uživatelé oceňují jeho flexibilitu a širokou škálu konfiguračních možností, které jsou přístupné i bez hlubokých znalostí Linuxu.

Proxmox VE je také známý svou komunitou a širokou podporou. Jako open-source řešení má aktivní komunitu vývojářů a uživatelů, kteří neustále přispívají k jeho vývoji a zlepšování. To zahrnuje pravidelné aktualizace, bezpečnostní opravy a nové funkce, což udržuje Proxmox VE jako relevantní a konkurenceschopnou platformu v oblasti virtualizace.

V současnosti je Proxmox široce používán v mnoha odvětvích, od malých podniků až po velké korporace, díky své schopnosti efektivně spravovat rozsáhlé a komplexní virtualizované prostředí. Jeho kombinace výkonu, flexibility a nízkých nákladů činí z Proxmoxu atraktivní volbu pro mnoho organizací hledajících efektivní virtualizační řešení.
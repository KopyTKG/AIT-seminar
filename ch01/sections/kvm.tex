\section{KVM}
KVM (Kernel-based Virtual Machine) je open-source virtualizační technologie pro Linux, která umožňuje transformovat Linux na hypervisor typu 1. Tato technologie využívá virtualizační funkce moderních procesorů, jako jsou Intel VT-x a AMD-V, a integruje se přímo do Linuxového jádra. Zde je podrobnější pohled na to, jak KVM funguje:

% \subsection{Základní Principy}
KVM se stává součástí Linuxového jádra, což znamená, že každý Linuxový server s odpovídajícím jádrem a podporovaným hardwarem může fungovat jako hypervisor.
KVM přidává do jádra funkce pro správu a spouštění virtuálních strojů (VM), zatímco samotné VM jsou standardní Linuxové procesy, řízené plánovačem jádra.

% \subsection{Hardwarová Virtualizace}
KVM využívá hardwarovou podporu pro virtualizaci, která je poskytována moderními procesory. Tato hardwarová podpora umožňuje KVM efektivně spouštět izolované VM, každý s vlastním virtuálním hardwarem, včetně CPU, paměti, disků a síťových adaptérů.
Virtualizovaný hardware VM je emulován pomocí QEMU (Quick Emulator), který poskytuje emulaci zařízení pro VM.

% \subsection{Vytváření a Správa VM}
VM jsou v KVM vytvářeny a spravovány pomocí standardních nástrojů pro virtualizaci, jako je libvirt a jeho nástroj virsh, nebo grafická rozhraní jako virt-manager.
Každý VM je izolován od ostatních VM a od hostitelského systému, což zajišťuje bezpečnost a stabilitu.

% \subsection{Výkon a Škálovatelnost}
Díky hardwarové podpoře a úzké integraci s Linuxovým jádrem nabízí KVM vynikající výkon a škálovatelnost, což je ideální pro náročné serverové a cloudové prostředí.
KVM podporuje funkce jako NUMA (Non-Uniform Memory Access), což umožňuje efektivní využívání paměti v multiprocesorových systémech.

% \subsection{Bezpečnost a Izolace}
VM v KVM jsou dobře izolovány díky použití hardwarových funkcí virtualizace a bezpečnostních mechanismů Linuxu, jako jsou SELinux a cgroups.
Tato izolace pomáhá chránit před škodlivým softwarem a zajišťuje, že chyba nebo selhání jednoho VM neovlivní ostatní VM nebo hostitelský systém.

% \subsection{Live Migrace}
KVM podporuje live migraci, což umožňuje přesunutí běžících VM z jednoho fyzického serveru na druhý bez výpadku služby. Toto je klíčové pro údržbu a řízení zatížení v datových centrech.

% \subsection{Ekosystém a Integrace}
KVM je široce podporován v rámci Linuxové komunity a integruje se s mnoha dalšími technologiemi a nástroji, což umožňuje širokou škálu použití od jednoduchých virtuálních hostů až po komplexní cloudové řešení.
V souhrnu, KVM je výkonný a efektivní nástroj pro virtualizaci, který poskytuje robustní, bezpečnou a škálovatelnou platformu pro správu a provoz virtuálních strojů v Linuxovém prostředí.

\section{LXC}
LXC (Linux Containers) je technologie pro virtualizaci na úrovni operačního systému, která je implementována v Linuxovém jádře. LXC umožňuje uživatelům spouštět více izolovaných instancí operačního systému na jednom hostitelském systému. LXC poskytuje efektivní, bezpečnou a flexibilní platformu pro běh izolovaných Linuxových prostředí na jednom hostitelském systému, využívající přitom výhod Linuxového jádra a jeho schopností. Tato technologie je klíčová pro moderní cloudové a kontejnerové řešení.

% \subsection*{Základní Koncepty:}
% \begin{itemize}[label=]
     LXC využívá Linuxové namespaces pro izolaci aplikací a procesů. Namespaces oddělují systémové aspekty (jako jsou síť, uživatelé, souborový systém atd.), což umožňuje každému kontejneru mít vlastní izolované prostředí, které se jeví jako samostatný systém. Dále také používá Cgroups pro omezeni a monitorování zdrojů, které mohou být používány procesy. Pomocí cgroups může LXC kontrolovat a omezovat, kolik CPU, paměti, síťové kapacity a dalších zdrojů může každý kontejner využívat.
% \end{itemize}

% \subsection*{Izolace a Bezpečnost:}
% \begin{itemize}[label=]
    LXC poskytuje silnou izolaci mezi kontejnery a mezi kontejnery a hostitelským systémem. To je zásadní pro bezpečnost, protože procesy běžící v jednom kontejneru nemohou přistupovat k procesům nebo zdrojům jiných kontejnerů nebo hostitelského systému. Bezpečnostní funkce, jako jsou AppArmor nebo SELinux, mohou být použity pro další zabezpečení kontejnerů, například omezují, které systémové volání mohou kontejnery používat.
% \end{itemize}

% \subsection*{Architektura a Výkon:}
% \begin{itemize}[label=]
    Na rozdíl od plné virtualizace, jako je KVM nebo Xen, LXC neemuluje hardwarovou vrstvu. Místo toho všechny kontejnery sdílejí stejné jádro hostitelského systému. Tato "lehká" virtualizace vede k menší režii, vyššímu výkonu a rychlejšímu spouštění kontejnerů. Na příč sdilenému jádru kontejnery LXC mohou běžet různé distribuce Linuxu, což umožňuje flexibilní nasazení aplikací bez ohledu na distribuci hostitelského systému.
% \end{itemize}

% \subsection*{Správa a Použití:}
% \begin{itemize}[label=]
     LXC je vybaveno nástroji pro snadné vytváření, spouštění, zastavování a spravování kontejnerů. Tyto nástroje umožňují uživatelům efektivně spravovat životní cyklus kontejnerů. Mezi tito nástroje patří nástroje jako jsou například \textbf{lxc-create}, \textbf{lxc-start}, \textbf{lxc-stop} a \textbf{lxc-destroy}. Tyto nastroje jsou základnými funkcemi LXC pro tvorbu, spouštění, zastavení a smazání kontejnerů.

% \end{itemize}

% \subsection*{Integrace a Rozšiřitelnost:}
% \begin{itemize}[label=]
     LXC je často využíváno jako základ pro vyšší úrovni kontejnerových orchestrací a management nástrojů, jako je Docker a Kubernetes, i když tyto nástroje v současné době používají vlastní kontejnerové runtime. Prostřednictvím LXC API mohou vývojáři integrovat funkce kontejnerů do svých aplikací nebo vytvářet vlastní nástroje pro správu kontejnerů.
% \end{itemize}


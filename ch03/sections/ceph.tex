\section{Ceph}

Ceph je vysoce škálovatelné distribuované úložiště, které přináší významné výhody v oblasti ukládání dat. Jeho architektura je navržena tak, aby poskytovala vynikající odolnost, flexibilitu a škálovatelnost, což je důvod, proč je často volbou pro komplexní datová centra a cloudová prostředí.

Hlavní zaměření Ceph je na vysokou dostupnost a odolnost proti chybám. Data jsou distribuována a replikována napříč různými uzly, což zajišťuje, že v případě selhání jednoho uzlu jsou data stále dostupná a chráněna. Toto automatické replikování a samoopravné schopnosti Cephu znamenají, že data jsou neustále chráněna a dostupná, což je klíčové pro podnikové aplikace a služby vyžadující nepřetržitý provoz.

Jednou z nejvýznamnějších vlastností Cephu je jeho schopnost škálovat bez výrazných omezení. Ceph může být rozšířen o další servery a úložné jednotky s minimální konfigurací a bez nutnosti přerušení služeb. Tato škálovatelnost umožňuje organizacím růst podle svých potřeb, přičemž výkon systému zůstává konzistentní a spolehlivý.

Vyskutuje se zde také podpora různých typů úložiště, včetně blokového úložiště (Ceph Block Device), souborového systému (CephFS) a objektového úložiště (Ceph Object Storage). Tato flexibilita umožňuje použití Cephu pro širokou škálu úložných potřeb, od běžných souborových serverů až po velkokapacitní objektové úložiště, což je ideální pro cloudové služby a aplikace.

Ceph se dobře integruje s různými open-source projekty a je široce podporován v cloudových a virtualizačních řešeních, jako je OpenStack a Proxmox. Jeho otevřená architektura a komunitní podpora zajišťují pravidelné aktualizace a vylepšení, což udržuje Ceph na špici technologického pokroku.

Použití Cephu může být také ekonomicky výhodné, protože je založeno na open-source modelu a umožňuje využití běžného hardwaru. Toto snižuje náklady spojené s proprietárními úložnými řešeními a zároveň poskytuje robustní a vysoce dostupné úložiště.

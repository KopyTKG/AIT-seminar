\chapter{Porovnaní Proxmox a VMware}
Porovnání Proxmox VE a VMware ESXi je klíčové pro pochopení, jak tyto dvě platformy pro virtualizaci nabízí různé možnosti a výhody pro správu datových center.

Proxmox VE je open-source řešení, které poskytuje virtualizaci pomocí KVM a kontejnerizaci prostřednictvím LXC. Je známé svou flexibilitou a širokým spektrem funkcí. Na druhé straně, VMware ESXi je průmyslovým standardem v oblasti virtualizace, nabízející robustní a stabilní platformu s širokou podporou pro různé operační systémy a aplikace.

Co se týče uživatelského rozhraní a správy, Proxmox VE nabízí webové rozhraní, které je intuitivní a snadno se ovládá, a umožňuje správu VM a kontejnerů, clustrování, zálohování a další. VMware ESXi naopak používá vSphere pro správu, což je mocný nástroj, ale může být složitější na naučení a používání, zejména pro nové uživatele.

Ve výkonu a škálovatelnosti je Proxmox VE efektivní a škálovatelný, umožňuje snadnou správu zdrojů a jejich rozšiřování. VMware ESXi je známý vysokým výkonem a škálovatelností, což je důvod, proč je preferován v náročnějších podnikových prostředích.

Podpora a komunita jsou také důležité aspekty. Proxmox VE má silnou a aktivní open-source komunitu, která poskytuje bezplatnou podporu a bohaté zdroje pro uživatele. Na druhé straně, VMware má rozsáhlou profesionální podporu a rozsáhlou uživatelskou základnu, což je klíčové pro podniky vyžadující garantovanou podporu a služby.

Licencování a náklady jsou další důležité faktory. Proxmox VE je bezplatný, s volitelným předplatným pro profesionální podporu a přístup k enterprise úložištím. Naproti tomu, VMware ESXi vyžaduje licenci, což může znamenat významné náklady, zejména ve větších nasazeních.

Proxmox VE nabízí vysokou úroveň přizpůsobitelnosti a množství doplňků, včetně silné podpory pro zálohování a obnovu. VMware poskytuje širokou škálu funkcí a integrací s dalšími produkty VMware, což z něj činí komplexní řešení pro podniková prostředí.

Migrace a flexibilita jsou další klíčové aspekty. Proxmox VE umožňuje snadnou migraci VM a kontejnerů mezi hosty a podporuje živou migraci. VMware ESXi také podporuje živou migraci VM, ale může být více omezený vzhledem k hardwarovým a licenčním požadavkům.

Bezpečnost je také důležitá. Proxmox VE nabízí dobré bezpečnostní funkce, ale jako open-source řešení může vyžadovat více úsilí pro nastavení a udržování bezpečnosti. VMware ESXi je široce považován za velmi bezpečnou platformu s pravidelnými aktualizacemi a siln

ou bezpečnostní architekturou.

Výběr mezi Proxmox VE a VMware ESXi závisí na konkrétních potřebách organizace. Pro malé a střední podniky nebo pro ty, kteří preferují open-source řešení, může být Proxmox VE vhodnější. Pro velké podniky, které vyžadují rozsáhlé funkce, vysokou stabilitu a profesionální podporu, je často preferován VMware ESXi. Obě platformy nabízejí silné řešení pro virtualizaci s různými výhodami a omezeními.
\chapter{Jádro}

\section{Uživatelské rozhraní a správa}
Proxmox VE se vyznačuje výkonným webovým rozhraním, které umožňuje snadnou správu virtuálních strojů, kontejnerů, sítí a úložiště. Rozhraní nabízí přehledný dashboard, detailní statistiky v reálném čase, logy a možnost konfigurace různých aspektů virtualizovaného prostředí.


\section{Zálohování a obnova}
Proxmox obsahuje integrované nástroje pro zálohování a obnovu virtuálních strojů a kontejnerů. Tyto nástroje umožňují plánovat pravidelné zálohy, což zajišťuje ochranu dat a usnadňuje obnovu v případě selhání.

\section{Síťové funkce}
Proxmox nabízí pokročilé síťové možnosti, včetně podpory VLAN, bondingu síťových rozhraní a firewallových funkcí. Umožňuje také snadnou integraci s externími síťovými službami a zařízeními.

\section{Migrace a flexibilita}
Live migrace umožňuje přesun virtuálních strojů mezi hostitelskými servery bez výpadku služby. Toto je klíčové pro údržbu hardwaru a optimalizaci zdrojů bez narušení provozu.

\section{Rozšiřitelnost a API}
Proxmox VE je vybaven REST API, které umožňuje automatizaci správy prostřednictvím skriptů a integraci s dalšími systémy. Komunita také poskytuje řadu pluginů a rozšíření pro další funkčnost.

\section{Bezpečnost}
Bezpečnostní funkce zahrnují šifrování na úrovni disku, podporu pro různé autentizační back-endy (LDAP, Active Directory, atd.) a izolaci prostředí prostřednictvím virtuáln